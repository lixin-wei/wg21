% ********************************* HEADERS ***********************************
\documentclass{article}
\usepackage[top=.75in, bottom=.75in, left=.50in,right=.50in]{geometry}
\usepackage{fancyhdr}
\usepackage{titling}
\pagestyle{fancy}
\lhead{Remove return type deduction in std::apply}
\rhead{\thepage}
\usepackage{ulem}
\usepackage{enumitem}
\usepackage{color}
\DeclareRobustCommand{\hlgreen}[1]{{\sethlcolor{green}\hl{#1}}}
\newcommand*\justify{%
  \fontdimen2\font=0.4em% interword space
  \fontdimen3\font=0.2em% interword stretch
  \fontdimen4\font=0.1em% interword shrink
  \fontdimen7\font=0.1em% extra space
  \hyphenchar\font=`\-% allowing hyphenation
}
\usepackage{graphicx}
\usepackage{listings}
\lstset{escapechar={|}}
\lstset{escapeinside={(*@}{@*)}}
\usepackage[utf8]{inputenc}
\usepackage{csquotes}
\lstset{
  % language=C++,
  showstringspaces=false,
  basicstyle={\small\ttfamily},
  numberstyle=\tiny\color{gray},
  keywordstyle=\color{blue},
  commentstyle=\color{dkgreen},
  stringstyle=\color{dkgreen},
}
\usepackage[colorlinks,urlcolor={blue}]{hyperref}
%\setlength{\parskip}{1em}
\usepackage{parskip}
\usepackage{indentfirst}
\setlength{\droptitle}{-4em}
\usepackage{soul}
\usepackage{xcolor}
\definecolor{darkgreen}{rgb}{0.0, 0.5, 0.0}
\lstdefinestyle{base}{
  basicstyle={\textcolor{darkgreen}\small\ttfamily},
  showstringspaces=false,
  numberstyle=\tiny\color{gray},
  keywordstyle=\color{blue},
  commentstyle=\color{dkgreen},
  stringstyle=\color{dkgreen},
}
\usepackage{makecell}
\usepackage{floatrow}

% \usepackage{concmath}
% \usepackage[T1]{fontenc}
% ********************************* HEADERS ***********************************
\floatsetup[table]{capposition=top}

\begin{document}
\title{\textbf{Remove return type deduction in \texttt{std::apply}}}
\author{
  Aaryaman Sagar\\
  \href{mailto:aary@meta.com}{\texttt{aary@meta.com}}
  \and
  Eric Niebler\\
  \href{mailto:eniebler@nvidia.com}{\texttt{eniebler@nvidia.com}}
}
\date{June 16, 2025 \\ Document number: D1317R2 \\ Library Working Group}
\maketitle

\section{Introduction}

\begin{lstlisting}
#include <tuple>

template <class Func, class Tuple>
concept applicable =
  requires(Func&& func, Tuple&& args) {
    std::apply(std::forward<Func>(func), std::forward<Tuple>(args));
  };

int main() {
  auto func = [](){};
  auto args = std::make_tuple(1);

  static_assert(!applicable<decltype(func), decltype(args)>);
}
\end{lstlisting}

The code above should be well formed.  However, since \texttt{std::apply} uses
return type deduction to deduce the return type, we get a hard error as the
substitution is outside the immediate context of the template instantiation.

This paper proposes a new public trait instead of \texttt{decltype(auto)} in
the return type of of \texttt{std::apply}

\section{Impact on the standard}
This proposal is a pure library extension.

\section{\texttt{std::apply\_result}}
\texttt{std::apply\_result} (and the corresponding alias
\texttt{std::apply\_result\_t}) is the proposed trait that should be used in
the return type of \texttt{std::apply}.  With the new declaration being:

\begin{lstlisting}
template <class F, class Tuple>
constexpr std::apply_result_t<F, Tuple> apply(F&& f, Tuple&& t);
\end{lstlisting}

This fixes hard errors originating from code that tries to employ commonly-used
SFINAE patterns with \texttt{std::apply} that could have otherwise been
well-formed. It is backwards compatible with well-formed usecases of
\texttt{std::apply}

\section{Implementation}
\texttt{std::apply\_result} can be defined using the existing
\texttt{std::invoke\_result} trait to avoid duplication in implementations

\begin{lstlisting}
namespace std {
  // exposition only
  template <size_t I, class T>
    using (*@\textit{element-at}@*) = decltype(get<I>(declval<T>()));

  // exposition only
  template <class F, class T, std::size_t... I>
  constexpr auto (*@\textit{apply-impl}@*)(F&& f, T&& t, std::index_sequence<I...>)
      noexcept(is_nothrow_invocable_v<F, (*@\textit{element-at}@*)<I, T>...>)
      -> invoke_result_t<F, (*@\textit{element-at}@*)<I, T>...> {
    return invoke(std::forward<F>(f), get<I>(std::forward<T>(t))...);
  }

  template <class F, class Tuple>
  using apply_result_t = decltype((*@\textit{apply-impl}@*)(
      declval<F>(),
      declval<Tuple>(),
      make_index_sequence<tuple_size_v<remove_reference_t<Tuple>>>()));

  template <class F, class Tuple>
  struct (*@\textit{apply-result-impl}@*) {}; // exposition only

  template <class F, class Tuple>
    requires requires { typename apply_result_t<F, Tuple>; }
  struct (*@\textit{apply-result-impl}@*)<F, Tuple> {
    using type = std::apply_result_t<F, Tuple>;
  };

  template <class F, class Tuple>
  struct apply_result : (*@\textit{apply-result-impl}@*)<F, Tuple> {};
 }
\end{lstlisting}

\section{Proposed Wording}

\textcolor{blue}{[\textit{Editorial note:} Insert the following declarations
to the listing in [concepts.syn]. \textit{--- end note}]}

\subsection*{18.3 Header \texttt{<concepts>} synopsis [concepts.syn]}
\begin{lstlisting}[style=base]
(*@\textit{// all freestanding}@*)
namespace std {
    (*@\textcolor{blue}{\textit{[...]}}@*)

  (*@\textit{// 18.7.7, concept strict\_weak\_order}@*)
  template<class R, class T, class U>
    concept strict_weak_order = (*@\textit{see below}@*);

  (*@\textcolor{darkgreen}{\texttt{\textit{// 18.7.8, concept applicable}}}@*)
  (*@\textcolor{darkgreen}{\texttt{template<class F, class Tuple>}}@*)
    (*@\textcolor{darkgreen}{\texttt{concept applicable = \textit{see below};}}@*)

  (*@\textcolor{darkgreen}{\texttt{\textit{// 18.7.9, concept regular\_applicable}}}@*)
  (*@\textcolor{darkgreen}{\texttt{template<class F, class Tuple>}}@*)
    (*@\textcolor{darkgreen}{\texttt{concept regular\_applicable = \textit{see below};}}@*)
}
\end{lstlisting}

\textcolor{blue}{[\textit{Editorial note:} Add the following two subsections at
the end of [concepts.callable], right after [concept.strictweakorder].
\textit{--- end note}]}

\begin{color}{darkgreen}
\subsection*{18.7.8 Concept \texttt{applicable} [concept.applicable]}
\begin{itemize}
\item The \texttt{applicable} concept specifies a relationship between a
callable type ([func.def]) \texttt{F} and a \texttt{\textit{tuple-like}} type ([tuple.like])
\texttt{Tuple} that can be evaluated by the library function \texttt{apply} ([tuple.apply]).

\begin{lstlisting}[style=base]
  template<class F, class Tuple>
    concept applicable = requires(F&& f, Tuple&& tup) {
      apply(std::forward<F>(f), std::forward<Tuple>(tup)); (*@\textit{// not required to be}@*)
                                                           (*@\textit{// equality-preserving}@*)
    };
\end{lstlisting}

\item {[}\textit{Example 1 }: A function that generates random numbers can model
\texttt{applicable}, since the \texttt{apply} function call expression is not required to be
equality-preserving ([concepts.equality]). \textit{--- end example}{]}
\end{itemize}


\subsection*{18.7.9 Concept \texttt{regular\_applicable} [concept.regularapplicable]}

\begin{lstlisting}[style=base]
template<class F, class Tuple>
  concept regular_applicable = applicable<F, Tuple>;
\end{lstlisting}

\begin{itemize}
\item The \texttt{apply} function call expression shall be equality-preserving
  ([concepts.equality]) and shall not modify the function object or the elements
  of the tuple.

  {[}\textit{Note 1}: This requirement supersedes the annotation in the definition
  of \texttt{applicable}. \textit{--- end note}{]}

\item {[}\textit{Example 1}: A random number generator does not model
  \texttt{regular\_applicable}. \textit{--- end example}{]}

\item {[}\textit{Note 2}: The distinction between \texttt{applicable} and
  \texttt{regular\_applicable} is purely semantic. \textit{--- end note}{]}
\end{itemize}
\end{color}


\textcolor{blue}{[\textit{Editorial note:} Add the following to the listing in [meta.type.synop]. \textit{--- end note}]}

\subsection*{21.3.3 [meta.type.synop]}
\begin{lstlisting}[style=base]
(*@\textit{// all freestanding}@*)
namespace std {
    (*@\textcolor{blue}{\textit{[...]}}@*)

  (*@\textit{// 21.3.7, type relations}@*)
    (*@\textcolor{blue}{\textit{[...]}}@*)
  template<class Fn, class... ArgTypes> struct is_invocable;
  template<class R, class Fn, class... ArgTypes> struct is_invocable_r;

  template<class Fn, class... ArgTypes> struct is_nothrow_invocable;
  template<class R, class Fn, class... ArgTypes> struct is_nothrow_invocable_r;

  (*@\textcolor{darkgreen}{\texttt{template<class Fn, class Tuple>\ struct is\_applicable;}}@*)

  (*@\textcolor{darkgreen}{\texttt{template<class Fn, class Tuple>\ struct is\_nothrow\_applicable;}}@*)
  
  (*@\textit{// 21.3.8.2, const-volatile modifications}@*)
    (*@\textcolor{blue}{\textit{[...]}}@*)

  (*@\textit{// 21.3.8.7, other transformations}@*)
    (*@\textcolor{blue}{\textit{[...]}}@*)
  template<class T> struct underlying_type;
  template<class Fn, class... ArgTypes> struct invoke_result;
  (*@\textcolor{darkgreen}{\texttt{template<class Fn, class Tuple>\ struct apply\_result;}}@*)
  template<class T> struct unwrap_reference;
  template<class T> struct unwrap_ref_decay;
    (*@\textcolor{blue}{\textit{[...]}}@*)
  template<class Fn, class... ArgTypes>
    using invoke_result_t = typename invoke_result<Fn, ArgTypes...>::type;
  (*@\textcolor{darkgreen}{\texttt{template<class Fn, class Tuple>}}@*)
    (*@\textcolor{darkgreen}{\texttt{using apply\_result\_t = typename apply\_result<Fn, Tuple>::type;}}@*)
  template<class T>
    using unwrap_reference_t = typename unwrap_reference<T>::type;
    (*@\textcolor{blue}{\textit{[...]}}@*)

  (*@\textit{// 21.3.7, type relations}@*)
  (*@\textcolor{blue}{... as before...}@*)
  template<class R, class Fn, class... ArgTypes>
    constexpr bool is_nothrow_invocable_r_v
      = is_nothrow_invocable_r<R, Fn, ArgTypes...>::value;
  (*@\textcolor{darkgreen}{\texttt{template<class Fn, class Tuple>}}@*)
    (*@\textcolor{darkgreen}{\texttt{inline constexpr bool is\_applicable\_v = is\_applicable<Fn, Tuple>::value;}}@*)
  (*@\textcolor{darkgreen}{\texttt{template<class Fn, class Tuple>}}@*)
    (*@\textcolor{darkgreen}{\texttt{inline constexpr bool is\_nothrow\_applicable\_v = is\_nothrow\_applicable<Fn, Tuple>::value;}}@*)

  (*@\textit{// 21.3.9, logical operator traits}@*)
    (*@\textcolor{blue}{\textit{[...]}}@*)
}
\end{lstlisting}


\textcolor{blue}{[\textit{Editorial note:} Change [meta.rel] as follows. \textit{--- end note}]}

\begin{enumerate}
\item The templates specified in Table 49 may be used to query
      relationships between types at compile time.

\item Each of these templates shall be a \textit{Cpp17BinaryTypeTrait}
      [meta.rqmts] with a base characteristic of \texttt{true\_type} if
      the corresponding condition is \texttt{true}, otherwise
      \texttt{false\_type}.

\begin{color}{darkgreen}
      \item Let \texttt{\textit{ELEMS-OF}(T)} be the parameter pack
      \texttt{get<\textit{N}>(declval<T>())}, where \texttt{\textit{N}} is the pack of
      \texttt{size\_t} template arguments of the specialization of \texttt{index\_sequence}
      denoted by
      \verb!make_index_sequence<tuple_size_v<remove_reference_t<T>>>!.
\end{color}
\end{enumerate}

\textcolor{blue}{[\textit{Editorial note:} At the end of Table 49, add the following. \textit{--- end note}]}
  
\begin{center}
  Table 49: Type relationship predicates
  \begin{tabular}[t]{ | p{6cm} p{7cm} p{5cm} | }
    \hline
    Template & Condition & Comments \\ 
    \hline\hline
    & ... as before ... & \\
    \hline
    \makecell[l]{\texttt{template<class R, class Fn,} \\
              \texttt{\ \ \ \ \ \ \ \ \ class... ArgTypes>} \\
              \texttt{struct is\_nothrow\_invocable\_r;}} &
    \makecell[l]{\texttt{is\_invocable\_r\_v<R, Fn, ArgTypes...>} \\
              is \texttt{true} and the expression \texttt{\textit{INVOKE}<R>(}\\
              \texttt{declval<Fn>(), declval<ArgTypes>()...)} \\
              is known not to throw any exceptions \\
              ([expr.unary.noexcept]).} & 
    \makecell[l]{\texttt{Fn}, \texttt{R}, and all types in the \\
              template parameter pack \\
              \texttt{ArgTypes} shall be \\
              complete types, \texttt{\textit{cv} void}, or \\
              arrays of unknown bound.} \\
    \hline
    \textcolor{darkgreen}{
      \makecell[l]{\texttt{template<class Fn,} \\
                \texttt{\ \ \ \ \ \ \ \ \ class Tuple>} \\
                \texttt{struct is\_applicable;}}} &
    \textcolor{darkgreen}{
      \makecell[l]{\texttt{\textit{tuple-like}<Tuple>} is \texttt{true} and \\
                the expression \texttt{\textit{INVOKE}(declval<Fn>(),} \\
                \texttt{\textit{ELEMS-OF}(Tuple)...)} is well-formed \\
                when treated as an unevaluated operand.}} &
    \textcolor{darkgreen}{
      \makecell[l]{\texttt{Fn} and \texttt{Tuple} shall be complete \\
                    types, \texttt{\textit{cv} void}, or arrays of \\
                    unknown bound.}} \\
    \hline
    \textcolor{darkgreen}{
      \makecell[l]{\texttt{template<class Fn,} \\
                \texttt{\ \ \ \ \ \ \ \ \ class Tuple>} \\
                \texttt{struct is\_nothrow\_applicable;}}} &
    \textcolor{darkgreen}{
      \makecell[l]{\texttt{is\_applicable\_v<Fn, Tuple>} is \texttt{true} \\
                and the expression \texttt{\textit{INVOKE}(declval<Fn>(),} \\
                \texttt{\textit{ELEMS-OF}(Tuple)...)} is known not to throw \\
                any exceptions ([expr.unary.noexcept]).}} &
    \textcolor{darkgreen}{
      \makecell[l]{\texttt{Fn} and \texttt{Tuple} shall be complete \\
                    types, \texttt{\textit{cv} void}, or arrays of \\
                    unknown bound.}} \\
    \hline
  \end{tabular}
\end{center}

\textcolor{blue}{[\textit{Editorial note:} Change [meta.trans.other] as follows. \textit{--- end note}]}

\subsection*{21.3.8.7 Other transformations [meta.trans.other]}

\begin{enumerate}
\item The templates specified in Table 55 perform other modifications of a type.
\end{enumerate}

\textcolor{blue}{[\textit{Editorial note:} Add the following to Table 55. \textit{--- end note}]}
  
\begin{center}
  Table 55 — Other transformations [tab:meta.trans.other]
  \begin{tabular}[t]{ | p{6cm} p{12cm} | }
    \hline
    Template & Comments \\ 
    \hline\hline
    & \textcolor{blue}{\textit{... as before ...}} \\
    \hline
    \makecell[l]{\texttt{template<class Fn} \\
                 \texttt{\ \ \ \ \ \ \ \ \ class... ArgTypes>} \\
                 \texttt{struct invoke\_result;}} &
    \makecell[l]{If the expression \texttt{\textit{INVOKE}(declval<Fn>(),} \\
                 \texttt{declval<ArgTypes>()...)} (22.10.4) is well-formed when treated as \\
                 an unevaluated operand (7.2.3), the member typedef \texttt{type} denotes \\
                 the type \texttt{decltype(\textit{INVOKE}(declval<Fn>(),} \\
                 \texttt{declval<ArgTypes>()...))}; otherwise, there shall be no member \\
                 \texttt{type}. Access checking is performed as if in a context unrelated to \texttt{Fn} \\
                 and \texttt{ArgTypes}. Only the validity of the immediate context of the \\
                 expression is considered. \\
                 $[$\textit{Note 2}: The compilation of the expression can result in side effects such as \\
                 the instantiation of class template specializations and function template \\
                 specializations, the generation of implicitly-defined functions, and so on. \\
                 Such side effects are not in the ``immediate context'' and can result in the \\
                 program being ill-formed. --- \textit{end note}$]$ \\
                 \textit{Preconditions}: \texttt{Fn} and all types in the template parameter pack \\
                 \texttt{ArgTypes} are complete types, \texttt{\textit{cv} void}, or arrays of unknown bound.
                 } \\
    \hline
    \textcolor{darkgreen}{
      \makecell[l]{\texttt{template<class Fn} \\
                  \texttt{\ \ \ \ \ \ \ \ \ class Tuple>} \\
                  \texttt{struct apply\_result;}}} &
    \textcolor{darkgreen}{
      \makecell[l]{If the expression \texttt{\textit{INVOKE}(declval<Fn>(), \textit{ELEMS-OF}(Tuple)...)} \\
                  (22.10.4) is well-formed when treated as \\
                  an unevaluated operand (7.2.3), the member typedef \texttt{type} denotes \\
                  the type \texttt{decltype(\textit{INVOKE}(declval<Fn>(),} \\
                  \texttt{\textit{ELEMS-OF}(Tuple)...))}; otherwise, there shall be no member \\
                  \texttt{type}. Access checking is performed as if in a context unrelated to \texttt{Fn} \\
                  and \texttt{Tuple}. Only the validity of the immediate context of the \\
                  expression is considered. \\
                  $[$\textit{Note 3}: The compilation of the expression can result in side effects such as \\
                  the instantiation of class template specializations and function template \\
                  specializations, the generation of implicitly-defined functions, and so on. \\
                  Such side effects are not in the ``immediate context'' and can result in the \\
                  program being ill-formed. --- \textit{end note}$]$ \\
                  \textit{Preconditions}: \texttt{Fn} and \texttt{Tuple} are complete types, \texttt{\textit{cv} void}, or arrays\\
                  of unknown bound.}} \\
    \hline
    \makecell[l]{\texttt{template<class T>} \\
                 \texttt{struct unwrap\_reference;}} &
    \makecell[l]{If \texttt{T} is a specialization \texttt{reference\_wrapper<X>} for some type \texttt{X}, the \\
                 member typedef type of \texttt{unwrap\_reference<T>} denotes \texttt{X\&}, \\
                 otherwise type denotes \texttt{T}.} \\
    \hline
    & \textcolor{blue}{\textit{... as before ...}} \\
    \hline
  \end{tabular}
\end{center}


\textcolor{blue}{[\textit{Editorial note:} Change the listing in [tuple.syn] as follows. \textit{--- end note}]}

\subsection*{Section 23.5.2 (\texttt{[tuple.syn]})}
\begin{lstlisting}[style=base]
    (*@\textcolor{blue}{\textit{[...]}}@*)

  template<(*@\textit{tuple-like}@*)... Tuples>
    constexpr tuple<CTypes...> tuple_cat(Tuples&&...);

  (*@\textit{// 23.5.3.5, calling a function with a tuple of arguments}@*)
  template<class F, (*@\textit{tuple-like}@*) Tuple>
    constexpr (*@\textcolor{red}{\st{decltype(auto)}}\textcolor{darkgreen}{\texttt{apply\_result\_t<F, Tuple>}}@*) apply(F&& f, Tuple&& t)
      noexcept((*@\textcolor{red}{\st{\textit{see below}}}\textcolor{darkgreen}{\texttt{is\_nothrow\_applicable\_v<F, Tuple>}}@*));

    (*@\textcolor{blue}{\textit{[...]}}@*)
\end{lstlisting}

\textcolor{blue}{[\textit{Editorial note:} Change [tuple.apply] as follows. \textit{--- end note}]}

\subsection*{22.4.6. Calling a function with a \texttt{tuple} of arguments [tuple.apply]}
\begin{lstlisting}
template<class F, (*@\textit{tuple-like}@*) Tuple>
  constexpr (*@\textcolor{red}{\st{decltype(auto)}}\textcolor{darkgreen}{\texttt{apply\_result\_t<F, Tuple>}}@*) apply(F&& f, Tuple&& t)
    noexcept((*@\textcolor{red}{\st{\textit{see below}}}\textcolor{darkgreen}{\texttt{is\_nothrow\_applicable\_v<F, Tuple>}}@*));
\end{lstlisting}

\begin{enumerate}
\item \textit{Effects:} Given the exposition-only function template:

\begin{lstlisting}[style=base]
namespace std {
  template<class F, (*@\textit{tuple-like}@*) Tuple, size_t... I>
  constexpr decltype(auto) (*@\textit{apply-impl}@*)(F&& f, T&& t, index_sequence<I...>) {
                                                                  (*@\textit{// exposition only}@*)
    return (*@\textit{INVOKE}@*)(std::forward<F>(f), get<I>(std::forward<T>(t))...);
                                                                  (*@\textit{// see [func.require]}@*)
  }
}
\end{lstlisting}

Equivalent to:

\begin{lstlisting}[style=base]
  return (*@\textit{apply-impl}@*)(std::forward<F>(f), std::forward<Tuple>(t),
                  make_index_sequence<tuple_size_v<remove_reference_t<Tuple>>>{});
\end{lstlisting}

\begin{color}{red}
\item[\st{2.}] \sout{\textit{Remarks:} Let \texttt{I} be the pack
      \texttt{0, 1, ... (tuple\_size\_v<remove\_reference\_t<Tuple>{}> - 1)}. The
      exception specification is equivalent to:}

\begin{lstlisting}[style=base]
  (*@\sout{noexcept(invoke(std::forward<F>(f), get<I>(std::forward<Tuple>(t))...))}@*)
\end{lstlisting}
\end{color}
\end{enumerate}


\end{document}
